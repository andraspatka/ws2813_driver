\tab Ennek a modulnak a feladata a LED-fűzérek összekötése a BRAM modulokkal. Egy generikus változó által lehet definiálni, hogy hány \textbf{WS2813\_Controller} szükséges a modulban (hány LED-fűzért kell megvezérelni).
A \textbf{WS2813\_Controller} modulok automatikusan lesznek kigenerálva, a generikus változó függvényében, viszont a BRAM modulokat muszáj kézzel kigenerálni és beilleszteni az adott modulba a megfelelő helyre.

\subsubsection{Elvégzendő RT műveletek azonosítása}

\tab A modulnak a feladata már egyszerű, csak össze kell kösse a már meglévő modulokat, ezért kevés logikát is tartalmaz, kevés RT művelet azonosítható. 
A modulnak tudnia kell, hogy a \textbf{WS2813\_Controller} moduljai mikor fejezték be a küldést. Ezért szükséges ezeknek a moduloknak a \textbf{done} jelei között egy logikai ÉS művelet elvégzése.

\tab RT művelet: $done \Leftarrow Controller\_done(0) \& ... \& Controller\_done(n - 1)$


\subsubsection{Adatfüggőségek identifikálása}

\tab Mivel nagyon egyszerű RT műveletekkel meg lehet oldani az adott feladatot, nem merűlnek fel adatfüggőségek.


\subsubsection{Célregiszterek azonosítása}

\tab Nincs szükség külön célregiszterekre a feladat megoldása érdekében.


\subsubsection{Különböző fázisokban elvégzendő műveletek}

TODO


\subsubsection{Kapcsolási rajz}

TODO

\subsubsection{Állapotdiagram}

TODO