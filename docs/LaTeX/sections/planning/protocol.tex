\tab A LED-eket vezérlő áramkörök egymás után vannak bekötve úgy, hogy az egyik áramkörnek az adatkimenete a következő áramkörnek az adatbemenetét képzi.
Egyszálú az adatátvitel, fontos a protokoll betartása, ahhoz, hogy adatokat tudjanak megjelenni a LED-fűzéren.

\tab Amikor egy áramkör megkap egy 24 bit-es kódot, akkor ezt addig tárolja amíg más kódot nem kap, vagy a tápforrást el nem veszti.

\subsubsection{A 24 bit-es kód}

\tab A 24 bit-es kód a következőképpen kell kinézzen: 

\tab \textbf{8 bit GREEN} | \textbf{8 bit RED} | \textbf{8 bit BLUE}

\tab Az adatátvitel a következő sorrendben kell történjen: 
\begin{enumerate}
\item GREEN
\item RED
\item BLUE
\end{enumerate}

\subsubsection{Bit-ek küldési sorrendje}

\tab Az egyes byte-ok küldését úgy kell elvégezni, hogy az \textbf{MSB}-vel kell kezdeni és haladni az \textbf{LSB} fele. 
24 bit-es kód részletesebb felbontása: 
\begin{itemize}
\item \textit{G7 G6 G5 G4 G3 G2 G1 G0 | R7 R6 R5 R4 R3 R2 R1 R0 | B7 B6 B5 B4 B3 B2 B1 B0}
\end{itemize}

\tab A küldés a következő sorrendben kell elvégződjön (balról jobbra): 
\begin{itemize}
\item \textbf{G7 G6 G5 G4 G3 G2 G1 G0 | R7 R6 R5 R4 R3 R2 R1 R0 | B7 B6 B5 B4 B3 B2 B1 B0}
\end{itemize}

\subsubsection{Időzítések}

\tab Minden 24 bit-es adatátvitel után kell legalább \SI{50}{\micro\second}-ot várakozni, alacsony feszűltségen. Ez jelzi azt, hogy egy 24 bit-es blokk továbbítása megtörtént.

\tab Az egyes bit-ek átvitele a következőképp történik:

\begin{itemize}
\item Logikai 1-es
	\begin{itemize}
	\item \SI{0.8}{\micro\second}-ot magas feszűltségen
	\item \SI{0.45}{\micro\second}-ot alacson feszűltségen
	\end{itemize}
\item Logikai 0-ás
	\begin{itemize}
	\item \SI{0.4}{\micro\second}-ot magas feszűltségen
	\item \SI{0.85}{\micro\second}-ot alacson feszűltségen
	\end{itemize}
\item 24 bit-es adatblokk küldése után: 
	\begin{itemize}
		\item $ > \SI{50}{\micro\second}$-ot alacsony feszűltségen
	\end{itemize}
\end{itemize}

\tab A bit-ek továbbításánál egy +/- \SI{150}{\nano\second}-os eltérés megengedett.

\tab A várakozási értékek nem az adatlapból, hanem az \href{https://learn.adafruit.com/adafruit-neopixel-uberguide}{alábbi} útmutatóból lettek kivéve. 
Az útmutató szerint az adatlapban levő értékek rosszul vannak kiszámolva.

\tab Ha az útmutatóban megadott értékek nem megfelelő működéshez vezetnek, akkor az adatlapban levő értékek lesznek felhasználva. 
Ez a valós rendszeren levő tesztelés közben fog kiderülni.
