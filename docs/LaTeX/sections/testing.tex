\tab Működés közbeni tesztelés az FPGA-ba beépített ILA (Integrated Logic Analyser) és egy Tektronix logikai analizátorral volt elvégezve. A tesztelés beállításához (ILA) a TCL script-es megoldás\cite{brassai2018ukda_ila} volt felhasználva.
A Tektronix logikai analizátorral való teszteléssel a vezető tanár segített.

\subsection{WS2813\_Driver - ILA}

\tab Ennél a modulnál szükség van egy 24 bit-es bementre. Mivel annyi kapcsoló nem található a választott FPGA lapon, ezért a modul le lett egyszerűsítve, a tesztelés megkönnyebítéséért. 
24 bemenet helyett csak 4 lett felhasználva, ezek be is lettek konfigurálva az fpga négy kapcsolójára.

\tab Ezen kívül a done és a d\_out jelek egy-egy LED-re lettek kötve. A d\_out jelnek ledre való kötése később értelmetlennek bizonyult, mivel olyan gyors a váltakozás magas feszűltésgről alacsony feszűltségre
a küldés során, hogy a LED fel sem gyúl.

\tab A rendszer működés közbeni teszteléséhez szükséges volt egy trigger definiálásához: amikor a d\_out jel 1-esre vált, akkor kezdődjön a mintavételezés. Ez lehetővé teszi, hogy a küldés kezdetétől legyen a mintavételezés.

\tab FPGA állapota 0000 küldése esetén:

\includegraphics[scale=0.2]{fpga_0000.jpg}

\tab Jelek 0000 küldése esetén:

\includegraphics[scale=0.43]{ila_0000.PNG}

\tab Amint az FPGA-ról és a jelanalizálása közben is látható, az elküldött adat 0000. Ugyanakkor az is látható, hogy a d\_out jel annyi időt van magas és alacsony feszűltségen, amennyit a protokoll megkövetel.
Ezen a példán nem látszik, hogy a küldés az MSB-től (Most Significant Bit) lenne elkezdve. Ennek demonstrálásához egy "asszimetrikus" adat szükséges, mint például: 0010

\tab FGPA állapota 0010 küldése esetén:

\includegraphics[scale=0.2]{fpga_0010.jpg}

\tab Jelek 0010 küldése esetén:

\includegraphics[scale=0.40]{ila_0010.PNG}

\tab Látható, hogy a küldés az MSB-től van elkezdve. Ugyanakkor az is megfigyelhető, hogy a WS2813 protokollja be van tartva.

\subsection{Teljes rendszer működés közbeni tesztelése - Tektronix logikai analizátor}

\tab A laborban található Tektronix logikai analizátorral sikerült elvégezni a működés közbeni tesztelést. A tesztelés alatt látható volt, hogy a 24 bit-es blokkok
helyesen küldődnek el, ugyanakkor az egyes bitek időzítése is helyesen történik.

Egy 24 bites blokk:

\includegraphics[scale=0.4]{tek00000.png}

24 bites blokkok küldése és a közöttük levő késleltetés

\includegraphics[scale=0.4]{tek00003.png}